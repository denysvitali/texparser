% arara: pdflatex
% arara: bibtex
% arara: pdflatex
% arara: pdflatex
\documentclass{article}

\usepackage{lipsum}
\usepackage{amsmath}
\usepackage{pifont}
\usepackage{textcase}
\usepackage[colorlinks]{hyperref}

\title{Sample Article}
\author{Nicola Talbot\thanks{Some attribution}}

\newcommand*{\pfrac}[2]{\frac{\partial#1}{\partial#2}}
\newcommand*{\dpfrac}[3]{\frac{\partial^2#1}{\partial#2\partial#3}}

\renewcommand*{\vec}[1]{\boldsymbol{#1}}

\begin{document}
\maketitle

\tableofcontents

\begin{abstract}
\lipsum[1]
\end{abstract}

\section*{An Unnumbered Section}
\addcontentsline{toc}{section}{An Unnumbered Section}

\lipsum[4-5]

\section{Introduction}
\label{sec:intro}

This is an example section. An unnumbered equation:
\[
  f(\vec{a}; x) = \sum_{i=1}^n a_i x^i
\]

A numbered equation:
\begin{equation}
y = m x + c
\label{eq:mx+c}
\end{equation}

Another numbered equation:
\begin{equation}\label{eq:dp}
 \dpfrac{L}{\eta_j}{w_k}
 = \dpfrac{L}{w_k}{\eta_j}
 = \sum_{i=1}^n \pfrac{g(z_i)}{z_i}
   \phi_k(\vec{x}_i)\pfrac{z_i}{\eta_j}
 + (\hat{y}_i - y_i)\pfrac{\phi_k(\vec{x}_i)}{\eta_j}
\end{equation}

Oh, no! It's an \texttt{eqnarray}:
\begin{eqnarray}
f(x) &=& ax^2 +b\label{eq:f}\\
f'(x) &=& 2ax\label{eq:df}
\end{eqnarray}
and an \texttt{eqnarray*}:
\begin{eqnarray*}
\mathcal{I} & = & \int_0^\infty f(x)\,dx\\
g(x) & = & \prod_i h_i(x)
\end{eqnarray*}

Some \textbf{bold text}, \textit{italic text}, 
\textsf{sans-serif text}, \textsc{small caps}.
Here's a footnote.\footnote{Here's the footnote text.}

\section{Case-Changing}
\label{sec:casechange}

\TeX\ case-change: \uppercase{upper \lowercase{LOWER} MiXeD $\alpha \neq a$.}
\LaTeX\ case-change: \MakeUppercase{abc\ae\ \protect\( a = b
\protect\) and $\alpha \neq a$}.
\texttt{textcase.sty} case-change: \MakeTextUppercase{abc\ae\ \( a = b \) and 
$\alpha \neq a$ \NoCaseChange{No Case-Change}}.

\newcommand{\foo}{fooa}
\newcommand{\FOO}{foob}
Upper: \uppercase{\foo}. Lower: \lowercase{\FOO}.
TC: Upper: \MakeTextUppercase{\foo}. Lower:
\MakeTextLowercase{\FOO}.


\section{Another Section}

Some cross-references: section~\ref{sec:intro},
equation~\ref{eq:mx+c} and some
citations~\cite{article-full,incollection-full}
and~\cite[some text]{inproceedings-full}.

Some more references: equations~\ref{eq:f} and~\ref{eq:df}.

\lipsum[2-3]

\subsection{A Subsection}

\lipsum[6-7]

\section*{Another Unnumbered Section}
\addcontentsline{toc}{section}{Another Unnumbered Section}

Some symbols: \ding{67}, \ding{118}, \ding{161}

\lipsum[8-9]

\bibliographystyle{plain}
\bibliography{xampl}

\end{document}
