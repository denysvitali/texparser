% arara: pdflatex
% arara: pdflatex
\documentclass{article}

\usepackage[T1]{fontenc}
\usepackage{lipsum}
\usepackage{graphicx}
\usepackage{jmlrutils}

\title{Sample Document}
\author{Sample Author}

\begin{document}
\maketitle

\begin{abstract}
\lipsum[1]
\end{abstract}

\section{Introduction}

This is a sample article. \sectionref{sec:method} discusses
the method used. \equationref{eq:emc2} is an interesting
equation. Some other stuff is in \appendixref{apd:first}.
See also \figureref{fig:example1,fig:example2,fig:sub-a,fig:sub-b}.

\lipsum[1]

\section{Method}\label{sec:method}

\lipsum[2]

\begin{equation}\label{eq:emc2}
E = mc^{2}
\end{equation}

\lipsum[3]

\begin{figure}
\floatconts
 {fig:example1}% label
 {\caption{An example image}}% caption
 {\includegraphics[width=.5\linewidth]{example-image}}% contents
\end{figure}

\begin{figure}
\floatconts
 {fig:example2}% label
 {\caption{Another example image}}% caption
 {\includegraphics[width=.5\linewidth]{example-image-a}}% contents
\end{figure}

\begin{figure}
\floatconts
 {fig:example3}% label
 {\caption{An example image with sub-figures}}% caption
 {% contents
  \subfigure
   {\label{fig:sub-a}\includegraphics[width=.25\linewidth]{example-image-b}}%
  \quad
  \subfigure
   {\label{fig:sub-b}\includegraphics[width=.25\linewidth]{example-image-c}}%
 }
\end{figure}


\section{Sample}
A sample section.

\begin{theorem}
A sample theorem.
\begin{proof}
Sample proof.
\end{proof}
\end{theorem}

\appendix
\section{First Appendix}\label{apd:first}

\lipsum


\end{document}
